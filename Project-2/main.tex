\documentclass[oneside,12pt]{article}
\usepackage{indentfirst}
\usepackage{fullpage}
\usepackage{blindtext, graphicx}
\usepackage{graphicx}
\usepackage{lipsum}
\usepackage{booktabs}
\usepackage{caption}
\usepackage{tabularx}
\usepackage{amsmath}
\usepackage{tabu}
\usepackage{float}
\usepackage{url}
\usepackage{color}
\usepackage[colorlinks=true,linkcolor=black, urlcolor=blue]{hyperref}
\setlength{\tabcolsep}{20pt}

\begin{document}
\pagenumbering{gobble}
\vfill
\begin{titlepage}
\begin{center}
\line(1,0){400}\\
\vspace{2cm}
\LARGE\textbf{MIDDLE EAST TECHNICAL UNIVERSITY}\\
\vspace{0.5cm}
\LARGE\textbf{DEPARTMENT OF ELECTRICAL AND ELECTRONICS ENGINEERING}\\ 

\vspace{1.5cm}
\LARGE\textbf{EE564 Project \#2}\\
\vspace{0.5cm}
\today\\
\vspace{2cm}

\line(1,0){400}\\
\vspace{0.5cm}

\end{center}

\vspace{3cm}
\begin{flushleft}
%\begin{tabular}{lll}
\LARGE\textbf {Student Name:} Goksenin Hande BAYAZIT\\
\LARGE\textbf {Student Number:} 2093441\\
%\end{tabular}
\end{flushleft}
\end{titlepage}
\newpage
\pagenumbering{arabic}

\section{Introduction}

In the scope of this project, a PM synchronous machine with a diameter of 240 mm is designed, as a part of design of an integrated modular motor drive system. Considering the previous design with the larger outer diameter, in this iteration, aspect ratio will be increased as the diameter becomes smaller with same output power (and torque). Motor parameters will be analyzed in detail using both analytical calculation techniques and computational tools.


\section{Winding Design}



\section{Parameter Estimation}

\section{Detailed Analysis & Verification}

\end{document}