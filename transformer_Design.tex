\sloppy
\definecolor{lightgray}{gray}{0.5}
\setlength{\parindent}{0pt}

\subsection*{Code for Transformer Design and Optimization}   
    
\subsubsection*{Parameters}

\begin{verbatim}
B_op=1.5; %T
mu=1.83/800;
f=50; %Hz
P1= 500e3; %W
V1=34500; %V
V2=25000; %V
I1=P1/V1; %A
J=4; %A/mm^2
a_cable=I1/J; %mm^2
fprintf('Cable area should be around %d mm^2.\n',a_cable);

% This value is close to AWG11 size.

a_cable=4.1684; %mm^2
dia_cable=2.30378; %mm
res_cable=4.1328; %Ohms/km
\end{verbatim}

        \color{lightgray} \begin{verbatim}Cable area should be around 3.623188e+00 mm^2.
\end{verbatim} \color{black}
    

\subsubsection*{Sizing}

\begin{verbatim}
% Recalling the equation: V_ind=2*pi/sqrt(2)*f*B*A*N
% A*N is constant and they are dependent on each other. Optimum values of
% them will be found with an optimization parameter "k".

ff=0.5; %fill factor
dens_steel=7650; %kg/m^3
dens_copper=8940; %kg/m^3
core_loss_dens=0.77; %W/kg
price_steel=3; %$/kg
price_copper=7; %$/kg

i=1;
for k=5:15

    N1(:,i)=69*k;
    N2(:,i)=50*k;
    A(:,i)=V2*sqrt(2)/(2*pi*f*B_op*N1(:,i)); %m^2

    % Window area

    x1(:,i)=dia_cable*23/ff/1000; %m
    x2(:,i)=dia_cable*3*k/ff/1000; %m
    x3(:,i)=ceil(sqrt(A(:,i)*10000))/100; %m

    w1(:,i)=x1(:,i);
    w2(:,i)=2*x2(:,i)+0.03;

%     fprintf('Window area is %d m^2.\n',w1(:,i)*w2(:,i));

    % Overall dimensions

    e1(:,i)=w1(:,i)+2*x3(:,i);
    e2(:,i)=w2(:,i)+2*x3(:,i);

%     fprintf('Dimensions of the transformer is %d x %d x %d m.\n',e1(:,i),e2(:,i),x3(:,i));

    vol(:,i)=(e1(:,i)*e2(:,i)-w1(:,i)*w2(:,i))*x3(:,i);

    m_steel(:,i)=dens_steel*vol(:,i);

%     fprintf('Steel mass is %d kg.\n',m_steel(:,i));

    core_loss(:,i)=core_loss_dens*m_steel(:,i);

%     fprintf('Core loss is %d Watts.\n',core_loss(:,i));

    % Cable length

    mean_length(:,i)=2*(x2(:,i)/2+x3(:,i))+pi*x3(:,i)*sqrt(2); %m
    l1(:,i)=mean_length(:,i)*N1(:,i);
    l2(:,i)=mean_length(:,i)*N2(:,i);

    r1(:,i)=l1(:,i)*res_cable/1000; %Ohms
    r2(:,i)=r1(:,i)*(N2(:,i)/N1(:,i))^2; %Ohms

    vol_copper(:,i)=2*l1(:,i)*a_cable/1000000; %m^3
    m_copper(:,i)=vol_copper(:,i)*dens_copper; %kg

%     fprintf('Copper mass is %d kg.\n',m_copper(:,i));

    copper_loss(:,i)=I1^2*r1(:,i)*2; %W

%     fprintf('Copper loss is %d Watts.\n',copper_loss(:,i));

    % Inductances

    % Assuming L1 and L2 are 0.02 pu;

    ind1=V1^2/(P1*2*pi*f)*0.02; %H
    ind2=ind1*(N2/N1)^2; %H

    Leff(:,i)=2*(w1(:,i)+w2(:,i)+2*x3(:,i)); %m

    ind_m(:,i)=N1(:,i).^2*mu*A(:,i)/Leff(:,i); %H

    % Efficiency

    eff(:,i)=P1/(P1+core_loss(:,i)+copper_loss(:,i))*100; %percent

    % Cost

    cost(:,i)=price_copper*m_copper(:,i)+price_steel*m_steel(:,i); %$

    %Unit price of electricity: 0.4482 TL = 0.1125 USD

    lost_power(:,i)=P1*(100-eff(:,i))/1000; %kW

    lost_energy(:,i)=lost_power(:,i)*24*365*20*0.1125; %$

    cost_actual(:,i)=cost(:,i)+lost_energy(:,i); %$


    i=i+1;
end
\end{verbatim}


\subsubsection*{Conclusion}

\begin{verbatim}
% Considering the optimizaton results, optimum case for the design seems to
% be i=6, k=10. Transformer parameters are as follows:

k=10;
i=6;

    N1(:,i)=69*k;
    N2(:,i)=50*k;
    A(:,i)=V2*sqrt(2)/(2*pi*f*B_op*N1(:,i)); %m^2



    % Window area

    x1(:,i)=dia_cable*23/ff/1000; %m
    x2(:,i)=dia_cable*3*k/ff/1000; %m
    x3(:,i)=ceil(sqrt(A(:,i)*10000))/100; %m

    w1(:,i)=x1(:,i);
    w2(:,i)=2*x2(:,i)+0.03;


    % Overall dimensions

    e1(:,i)=w1(:,i)+2*x3(:,i);
    e2(:,i)=w2(:,i)+2*x3(:,i);

    vol(:,i)=(e1(:,i)*e2(:,i)-w1(:,i)*w2(:,i))*x3(:,i);

    m_steel(:,i)=dens_steel*vol(:,i);

    core_loss(:,i)=core_loss_dens*m_steel(:,i);

    % Cable length

    mean_length(:,i)=2*(x2(:,i)/2+x3(:,i))+pi*x3(:,i)*sqrt(2); %m
    l1(:,i)=mean_length(:,i)*N1(:,i);
    l2(:,i)=mean_length(:,i)*N2(:,i);

    r1(:,i)=l1(:,i)*res_cable/1000; %Ohms
    r2(:,i)=r1(:,i)*(N2(:,i)/N1(:,i))^2; %Ohms

    vol_copper(:,i)=2*l1(:,i)*a_cable/1000000; %m^3
    m_copper(:,i)=vol_copper(:,i)*dens_copper; %kg

    copper_loss(:,i)=I1^2*r1(:,i)*2; %W

    % Inductances

    % Assuming L1 and L2 are 0.02 pu;

    ind1=V1^2/(P1*2*pi*f)*0.02; %H
    ind2=ind1*(N2/N1)^2; %H

    Leff(:,i)=2*(w1(:,i)+w2(:,i)+2*x3(:,i)); %m

    ind_m(:,i)=N1(:,i).^2*mu*A(:,i)/Leff(:,i); %H

    % Efficiency

    eff(:,i)=P1/(P1+core_loss(:,i)+copper_loss(:,i))*100; %percent

    % Cost

    cost(:,i)=price_copper*m_copper(:,i)+price_steel*m_steel(:,i); %$


    %Unit price of electricity: 0.4482 TL = 0.1125 USD

    lost_power(:,i)=P1*(100-eff(:,i))/1000; %kW

    lost_energy(:,i)=lost_power(:,i)*24*365*20*0.1125; %$

    cost_actual(:,i)=cost(:,i)+lost_energy(:,i); %$

    fprintf('Turns ratio is %d : %d. \n',N1(:,i),N2(:,i));

    fprintf('Window area is %d m^2.\n',w1(:,i)*w2(:,i));

    fprintf('Dimensions of the transformer is %d x %d x %d m.\n',e1(:,i),e2(:,i),x3(:,i));

    fprintf('Steel mass is %d kg.\n',m_steel(:,i));

    fprintf('Core loss is %d Watts.\n',core_loss(:,i));

    fprintf('R1= %d Ohms, R2= %d Ohms. \n',r1(:,i),r2(:,i));

    fprintf('Copper mass is %d kg.\n',m_copper(:,i));

    fprintf('Copper loss is %d Watts.\n',copper_loss(:,i));

    fprintf('L1= %d H, L2= %d H, Lm= %d H. \n',ind1,ind2,ind_m(:,i));

    fprintf('Efficiency is %d percent. \n',eff(:,i));

    fprintf('Material cost is %d USD. Lost money in 20 years is %d USD. \n',cost(:,i),cost_actual(:,i));
\end{verbatim}

        \color{lightgray} \begin{verbatim}Turns ratio is 690 : 500. 
Window area is 3.247608e-02 m^2.
Dimensions of the transformer is 7.659739e-01 x 9.664536e-01 x 3.300000e-01 m.
Steel mass is 1.786846e+03 kg.
Core loss is 1.375872e+03 Watts.
R1= 6.457173e+00 Ohms, R2= 3.390660e+00 Ohms. 
Copper mass is 1.164488e+02 kg.
Copper loss is 2.712528e+03 Watts.
L1= 1.515473e-01 H, L2= 7.957747e-02 H, Lm= 5.521107e+01 H. 
Efficiency is 9.918895e+01 percent. 
Material cost is 6.175681e+03 USD. Lost money in 20 years is 7.999056e+06 USD. 
\end{verbatim} \color{black}
    
  
