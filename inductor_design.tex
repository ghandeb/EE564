
% This LaTeX was auto-generated from MATLAB code.
% To make changes, update the MATLAB code and republish this document.

\documentclass{article}
\usepackage{graphicx}
\usepackage{color}

\sloppy
\definecolor{lightgray}{gray}{0.5}
\setlength{\parindent}{0pt}

\begin{document}

    
    
\section*{INDUCTOR DESIGN}

\begin{par}
EE564 Project\#1 Q1 (Analytical Calculations) by G. Hande Bayazit
\end{par} \vspace{1em}

\subsection*{Contents}

\begin{itemize}
\setlength{\itemsep}{-1ex}
   \item Parameters
   \item Part A
\end{itemize}


\subsection*{Parameters}

\begin{verbatim}
in_dia= 19; %mm
in_rad=in_dia/2*1e-3;
out_dia= 38.1; %mm
out_rad=out_dia/2*1e-3;
ht= 6.11; %mm

Leff=82.9; %mm
Aeff=56.1; %mm^2

N=15 ;% turns
I=0.3; % A

mu_r=5000;
mu_0=4*pi*1e-7;

AL=4400; %nH/turns^2
\end{verbatim}


\subsection*{Part A}

\begin{verbatim}
%
% # 1

reluctance=Leff*1e-3/(Aeff*1e-6*mu_r*mu_0);

L=N^2/reluctance; %H

fprintf('Inductance assuming homogeneous distribution is %d H. \n', L);

% # 2
% In this part, H.dl should be integrated over over the cross-section of
% coaxial circles with radii from in_dia/2 to out_dia/2.
% H*2*pi*(out_rad-in_rad)=N.I

index=linspace(in_rad,out_rad,500);
for i=1:numel(index)

    H(:,i)=N*I/(2*pi*index(:,i));
    B(:,i)=mu_0*mu_r*H(:,i); %T
    Phi(:,i)=B(:,i)*ht*1e-3*(out_rad-in_rad)/500; %Wb
end

tot_phi=0;

for k=1:500

   tot_phi=tot_phi+Phi(:,i);

end

L_2=tot_phi*N/I;

fprintf('Inductance assuming non-homogeneous distribution is %d H. \n', L_2);
%
% # 3_1
%
I_2=I*1.5;

H_2=N*I_2/(Leff*1e-3);

B_2=interp1(B_nl,H_nl,H_2);

phi_2=Aeff*1e-6*B_2;

L_3=N*phi_2/I_2;

fprintf('Inductance assuming homogeneous distribution & non-homogeneous material is %d H. \n', L_3);

%
% # 3_2
%


index=linspace(in_rad,out_rad,500);
for i=1:numel(index)

    H_3(:,i)=N*I_2/(2*pi*index(:,i));
    B_3(:,i)=interp1(B_nl,H_nl,H_3(:,i));
    Phi(:,i)=B_3(:,i)*ht*1e-3*(out_rad-in_rad)/500; %Wb
end

tot_phi=0;

for k=1:500

   tot_phi=tot_phi+Phi(:,i);

end

L_4=tot_phi*N/I_2;

fprintf('Inductance assuming non-homogeneous distribution & non-homogeneous material is %d H. \n', L_4);

%
% # 4
%

rel_gap=2e-3/(mu_0*Aeff*1e-6);
rel_core=(Leff-2)*1e-3/(Aeff*1e-6*mu_r*mu_0);
rel_tot=rel_gap+rel_core;

L_5=N^2/rel_tot;

fprintf('Inductance of the gapped core assuming homogeneous distribution is %d H. \n', L_5);

%
% # 5
%

% In this part, we may assume the fringing flux is considerable for an area
% of 2 mm (the length of airgap) to the left and to the right of the
% airgap. Therefore we may assume the equivalent reluctance of the magnetic
% circuit as R_core in series with (R_gap_left||R_gap||R_gap_right). This
% approach is not a reliable one, nevertheless it may give us an idea.

rel_gap_side=2e-3/(mu_0*2e-3*ht*1e-3);
rel_tot_2=rel_core+(1/rel_gap_side+1/rel_gap+1/rel_gap_side)^(-1);

L_6=N^2/rel_tot_2;

fprintf('Inductance of the gapped core including fringing flux is %d H. \n', L_6);
\end{verbatim}

        \color{lightgray} \begin{verbatim}Inductance assuming homogeneous distribution is 9.566889e-04 H. 
Inductance assuming non-homogeneous distribution is 6.891791e-04 H. 
Inductance assuming homogeneous distribution & non-homogeneous material is 5.383088e-04 H. 
Inductance assuming non-homogeneous distribution & non-homogeneous material is 5.027784e-04 H. 
Inductance of the gapped core assuming homogeneous distribution is 7.867304e-06 H. 
Inductance of the gapped core including fringing flux is 1.125535e-05 H. 
\end{verbatim} \color{black}
    


\end{document}
    
